\section{Knowing the Answer}

The analysis sketched in the previous section and developed in the
Appendix is one way of formalizing Schaffer's contextualist
intuition. We therefore agree with Schaffer that the truth-value
of knowing-\emph{wh} ascriptions will vary depending on the
background of possibilities that are considered relevant. However,
we remain skeptical on the idea that this should provide an
argument against the so-called reductive analysis of
knowing-\emph{wh}, or even against the soundness of the schema of
convergent knowledge. In this section, we discuss two ways in
which ``knowing the answer" can be ambiguous: depending on whether
the question is asked explicitly, and depending on whether one
knows a partial, or a total answer to the question. Along the way,
we examine another asymmetry in Schaffer's examples, which we
relate to the presupposition of alternative questions.

\subsection{Two senses of ``Knowing the answer"}

Regarding the schema of convergent knowledge, our main reason to
defend it is that if the background context $X$ is kept constant
in $(19)$, then convergent questions still yield equivalent
knowledge claims, and as explained in general $X$ is not simply a
function of the question asked. For instance, relative to $X=B\cup
F\cup J$, not only does S fail to know whether Bush or Ferrell is
on TV, but we have seen that S does not really know whether Bush
or Jackson is on TV either. Conversely, if one can indeed say ``S
knows whether it is Bush or Janet Jackson" and then ``but S does
not know whether it is Bush or Ferrell", then this is due
precisely to the fact that the background context $X$ differs in
the two cases.

As regards the reductive view of knowing-\emph{wh}, Schaffer
denies it by writing: ``there is more to knowledge-\emph{wh} than
knowing the proposition that just so happens to be the answer". We
do not quite agree with this view. For one thing, even a
context-dependent analysis of knowing-\emph{wh} like \ref{cd} is
expressed in terms of knowing-\emph{that}. More importantly, we
take Schaffer's example to be revealing rather of a form of
ambiguity in what ``knowing the answer" means in the first place.
There may be two senses of ``knowing the answer to the question
whether A or B": ``knowing whether A or B" \emph{before} the
question is asked explicitly by someone else, and ``knowing
whether A or B" \emph{after} the question has been asked
explicitly by someone else. A related point was made earlier by J.
Hawthorne, who writes that ``the very asking of a question may
provide one with new evidence regarding the subject
matter".\footnote{Hawthorne (2004: 78), quoted by Schaffer (2005),
fn. 22.} However, Schaffer dismisses it by considering that we can
make knowing-\emph{wh} ascriptions \emph{in absentia}, namely ``of
subjects who are miles away". We agree with him, but in our
opinion, knowledge ascriptions may differ depending on whether the
ascriber imagines: ``if I asked him whether A or B, how would he
answer?", or: ``what does he presently think about whether A or
B?".

Consider a pupil, on the way to her history exam, trying to
remember whether Napoleon I was born in 1768 or 1769. She cannot
remember whether Napoleon was born in 1768 or 1769. Then comes the
exam; one of the questions is: ``was Napoleon born in 1769, or in
1869?". S answers that Napoleon was born in 1769, because she
knows for sure that Napoleon was born before 1800. By doing so,
however, S makes the reasonable assumption that the question asked
contains the correct answer. In other words, S accommodates the
information contained in the question in order to restrict the
range of possibilities. Before any question was asked, however,
could one have said:

\ex. S knows whether Napoleon was born in 1769 or in 1869

This is not clear. For once again, S's knowledge that Napoleon was
not born in 1869 is not sufficient to settle the issue. After the
question was asked and S correctly answers ``1769", however, it
seems one can indeed say:

\ex. S knows the answer to the question whether Napoleon was born
in 1769 or 1869.

\noindent If so, the contrast suggests that ``knowing whether" and
``knowing the answer to the question whether" are not necessarily
synonymous, or more precisely, that ``knowing whether", just like
``knowing the answer", is potentially ambiguous, depending on
whether the subject's state of knowledge is taken to be restricted
to the alternatives raised by the questions or not.

%%%%%%%%%%%%%%%%%%%%%%%%% where things have to be changed

\subsection{Not knowing whether A or B}

The last example points toward another phenomenon. Before the
question was asked, it appears infelicitous to say:

\ex. S does not know whether Napoleon was born in 1769 or in 1869

For more generally, ``S does not know whether A or B" seems to
imply ``S does not know whether A and S does not know whether B".
However, in the situation under discussion, S knows that Napoleon
was not born in 1869. In the same way, Schaffer notes that the two
sentences:

\ex. \a. \label{phone} I forgot whether I left my keys on the
table, or by the phone. \b. \label{fridge} I forgot whether I left
my keys on the table, or in the fridge.

\noindent are clearly inequivalent in a context in which I
remember I did not leave my keys in the fridge, but don't remember
whether I left them on the table or by the phone, even if the two
embedded questions have the same answer (I left my keys on the
table). According to Schaffer, \ref{phone} is true and
\ref{fridge} is \emph{false} in that situation. Here, however, we
may wonder \ref{fridge} should not be considered \emph{undefined}
rather than false, namely whether the inequivalence does not
relate to a presupposition failure. Indeed, the reason \ref{phone}
and \ref{fridge} are not equivalent is that \ref{fridge} does not
satisfy the implication to ``I forgot whether I left my keys in
the fridge" in that context, whereas in the case of \ref{phone} I
am supposed to be uncertain about both options.

What is the nature of the inference from ``S does not know whether
A or B" to ``S does not know whether A and S does not know whether
B"? In the positive case, one can check that ``S knows whether A
or B" entails ``S knows whether A and S knows whether B". This
entailment is satisfied in the system presented in the Appendix,
irrespective of the contextual restrictions discussed so far. This
means that if S knows the complete answer to the alternative
question whether A or B, then S knows the complete answer to the
question whether A and to the question whether B. However, this
entailment is not preserved under negation in our system. From ``S
does not know whether A or B", we only predict ``S does not know
whether A \emph{or} S does not know whether B".

We can think of several ways to recover this implication, however.
The most obvious way would be to suppose that to know whether A or
B presupposes to know that exactly one of A or B holds. However,
this seems to us to be too restrictive. Let us imagine for
instance that John knows that Mary is both Italian and French.
Certainly, in such a case, ``John knows whether Mary is Italian or
French" is not false. The question is whether it should come out
as undefined rather than true. On our account, an alternative
question of the form ``Is Mary Italian, or French?" can be
answered by ``both" or ``neither" in principle, even if the
speaker happens to signal, by asking the question, that she
believes one and exactly one of the disjuncts to be
true.\footnote{In some contexts, one can ask: ``Is Mary Italian,
French, or both?" (and similarly with ``neither"). This suggests
that the default is to consider the answers ``both" or ``neither"
as unlikely or as excluded for alternative questions. But this may
be only pragmatic, and does not allow one to conclude that
semantically the two questions ``is Mary Italian or French?" and
``Is Mary Italian, or French, or both?" have different answerhood
conditions.} Consequently, we are inclined to say that ``John
knows whether Mary is Italian or French" would be true in this
case. By contrast, ``John does not know whether Mary is Italian or
French" would indeed sound odd if John is certain that Mary is not
Italian. But we can still capture this asymmetry between the
positive and the negative case if we make the weaker assumption
that:

\ex.\label{pr} ``S knows whether A or B" presupposes that S knows
(A and not B) or (B and not A), or S knows (A and B) or (neither A
nor B).

In other words, ``S knows whether A or B" is said to presuppose
that S knows it is exactly one of A or B, \emph{or that S knows it
is both or neither}. To go back to Schaffer's example, in a
situation in which S thinks it might be Ferrell or Bush on TV, ``S
knows whether Bush or Jackson is on TV" would come out as
undefined rather than false, because S's belief state overlaps on
the answers ``neither of them" and ``Bush and not Jackson".
Likewise,  assuming condition \ref{pr}, then one can check that
``S does not know whether A or B", \emph{together with its
presupposition}, entails ``S does not know whether A and S does
not know whether B". In Schaffer's scenario, ``S does not know
whether Bush or Jackson is on TV" will come out as undefined
rather than false, simply because S is certain it is not Jackson,
in the same way in which ``I don't know whether I left my keys on
the table or in the fridge" sounds odd in a situation in which I
am sure I did not leave my keys in the fridge. On the other hand,
``S does not know whether Bush or Ferrell is on TV" remains true
when adding the presupposition, because S's belief state is such
that S knows that exactly one of them is on TV.

By itself, the addition of this presupposition to an analysis like
\ref{cf} is therefore sufficient to account for the intuition that
``S knows whether Bush or Jackson is on TV" and ``S knows whether
Bush or Ferrell is on TV" are not semantically equivalent:
assuming \ref{pr}, and Groenendijk and Stokhof's semantics for
alternative questions, we predict the latter to be false, and the
former to be undefined. This is not quite Schaffer's intuition,
though, since Schaffer considers that ``S knows whether Bush or
Jackson is on TV" should come out \emph{true} in that context. The
system presented in the Appendix predicts that ``S knows whether
Bush or Jackson is on TV" may indeed come out as true or false,
depending on the alternatives considered. However, the system is
entirely bivalent, and does not incorporate a condition like
\ref{pr}. What happens if we add it? In that case, it is easy to
check that when restricted to the topics ``Bush" and ``Jackson",
the sentence remains true. When the set of topics integrates also
``Ferrell", then the sentence becomes undefined, rather than
false, just like its negation ``S does not know whether Bush or
Jackson is on TV".

Whether in the case under discussion the sentence should come out
as false rather than undefined is not a clear empirical matter in
our opinion. Either way, what matters on the present account is
that ``S knows whether A or B" does not come out true in all
contexts in which S can exclude B but is uncertain about A. It
comes true only in those contexts in which alternatives beside A
and B can be excluded. If we adopt \ref{pr}, however, a further
issue concerns the source of the presupposition. Here, \ref{pr}
gives us one way of deriving the inference from ``S does not not
know whether A or B" to ``S does not know whether A and S does not
know whether B". But it may not be the only way, nor the most
plausible one. Fundamentally, what \ref{pr} expresses is a
principle of symmetry in the agent's uncertainty concerning the
basic answers that can be given to the alternative question.
Whether such a principle is an instance of a more general
principle, or should be derived from the meaning of alternative
questions more specifically is an issue we shall leave for another
occasion.






%%%%%%%%%%%%%%%%%%%%%%%%%%%%%%%%%%%%%%

%More generally, the understanding of embedded questions will
%differ depending on the embedding verb and its presuppositional
%behavior.

\subsection{\emph{Could tell whether}}

We can note, finally, a contrast between ``know whether" and ``can
tell whether", or more clearly, ``could tell whether", for
alternative questions. Let us compare for instance:

\ex. \a.\label{know} S knows whether Napoleon was born in 1769 or
in 1869. \b.\label{tell} S could tell whether Napoleon was born in
1769 or in 1869.

Our intuition is that \ref{tell} is more appropriate than
\ref{know} for the kind of scenario that Schaffer considers, since
it refers more clearly to the fact that S is facing or might be
facing an explicit question. More precisely, ``S could tell
whether A or B" seems to mean: ``supposing A or B were actually
the case, S could tell which of them is", thereby justifying the
restriction of the alternatives to A and B.\footnote{Thus, in the
Bush-Ferrell scenario, for instance, it seems one can even say:
``S could tell whether it is Ferrell or Janet Jackson", meaning
that if it had to be only one of them, S could still rule out
Janet Jackson and choose Ferrell.} We do agree, however, that
``know" can be used to mean ``can tell", and likewise ``would
know" can be used to mean ``could tell". Nevertheless, this sense
of ``know" is compatible with weaker standards of knowledge. To be
sure, let us compare:


\ex.\a. (?) S does not know for sure whether Napoleon was born in
1768 or in 1769, but he knows whether Napoleon was born in 1769 or
in 1869. \b. S does not know for sure whether Napoleon was born in
1768 or in 1769, but he could tell whether Napoleon was born in
1769 or in 1869.

%\a. (??) S knows whether Napoleon was born in 1768 or in 1868,
%although he does not know for sure whether Napoleon was born in
%1768 or 1769. \b. S can tell whether Napoleon was born in 1768 or
%in 1868, although he does not know for sure whether Napoleon was
%born in 1768 or 1769.

%

Pierre can tell whether A or B if he knows and can provide the
partial answer ``not A", even if he does not know about B for
sure. Thus, ``can tell whether A or B" seems more adequate than
``know whether" to mean ``make the correct choice between A and
B". By contrast, for Pierre to know whether A or B, Pierre may
well have to know whether A \emph{and} to know whether B. Knowing
whether, we argue, is fundamentally ambiguous between these two
senses of knowing the answer in Schaffer's scenarios: knowing a
partial answer to the question, and being able to infer the actual
answer from that partial answer on the one hand, versus knowing
the true exhaustive answer directly on the other hand,
independently of the information brought up by the questioner. The
latter, of course, is more demanding than the former.\footnote{In
some cases, ``knowing the answer" may simply mean ``being able to
give the correct answer", whatever method has been used, even if S
has simply \emph{guessed} the correct answer. However, ``S could
tell whether A or B" seems to imply that S has a reliable method
to tell the correct answer.} As Schaffer writes, ``a student might
well know the answer when the options are easy". This, however, is
primarily ``knowledge in the eye of the ascriber": in other words,
this matches an externalist conception of knowledge, whereby the
attitude holder need not know that he knows in order to be
ascribed knowledge. Suppose the student answers ``1769" to the
question whether Napoleon was born in 1769 or 1869. By the
professor's standards, the student knows the answer. But does S
know that she knows the answer? This is less obvious, because S
may still doubt whether Napoleon was born in 1769, and fear the
professor might have devised a ``snare question". For S to know
that she knows the answer, S would have to know that Napoleon was
born in 1769 prior to the question being asked, without having to
trust the questioner on the correctness of the alternatives.
