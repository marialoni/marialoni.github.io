\documentclass[a4paper, 11pt]{article}
%\usepackage{covington}
\usepackage{linguex}
 
\newcommand{\bl}{$\bullet$\ }
\newcommand{\val}[1]{\mbox{ $[\![$  #1 $]\!]$}}
\newcommand{\bit}{\begin{itemize}} \newcommand{\eit}{\end{itemize}}
\newcommand{\ben}{\begin{enumerate}} \newcommand{\een}{\end{enumerate}}
\newcommand\notext[1]{}


% double brackets.
\newcommand{\lin}{\ensuremath{\lbrack\!\lbrack}}
\newcommand{\rin}{\ensuremath{\rbrack\!\rbrack}}


\gdef\bibname{Bibliography}
 

\title{Notes}

%\author{Maria Aloni }
 
\begin{document}
%\maketitle
%\date
\notext{In these notes, first I try to explain Schaffer's example in terms of presupposition accommodation, but I don't really get to a defendable analysis. Then I go back to our dynamic account  (not so bad after all) and try to repair the polar question problem. I wrote this a bit in a hurry, there might be typos and more serious mistakes in the formalization, if it is too hard to read let me know and I write more clearly.    
 
    
  \paragraph{Presupposition accommodation}
In this paragraph, we try to explain Schaffer's example   in terms of presupposition accommodation. Let us assume that whether A or B presupposes A or B (and not both). As far as I can see we could   get Schaffer's result only by {\it temporary intermediate accommodation}: the presupposition must be  accommodated in John's belief state rather than on the global state, and it is temporarily accommodated.


\ex. \a. John knows whether A or B.  \b.     $K_j +pres$ entails $Q(w)$ \b.   pres: $(A \vee B)  \wedge \neg (A \wedge B) $

We need temporary accommodation to account for the following:

\ex. \a. John knows whether A or B, but not whether A or C.  \b.    $K_j +pres_1$ entails $Q_1(w)$ \&        $K_j +pres_2$ entails $Q_2(w)$ 
\b. pres$_1$: $(A \vee B)  \wedge \neg (A \wedge B) $
\b. pres$_2$: $(A \vee C)  \wedge \neg (A \wedge C) $


Intermediate accommodation is dubious (see debate van der Sandt-Beaver), temporary accommodation is also unusual. So I find this solution pretty strange. But I might be very wrong. 


In any case,  we still want to explain the zebra sequence. 

\ex. 
\a. A or B?
\b. A
\c. But, A or C?
\d. I don't know.
\d. So, A or B? 
\d. I don't know, not B.

If at (a) you accommodate (A or B). Then to explain (d) we need to model some notion of  belief revision, unless again
 we assume some notion of temporary accommodation. This is more or less what Groenendijk does (in his notes on Schaffer),  he would analyze our zebra exchange as follows:
 \ex. \a. A or B?
 \b. If A or B, then A
 \c. But, A or C?
 \d. If A or C, then I don't know.
 \e. So A or B?
 \f. \# If A or B, then...
 
 
 Groenendijk can explain  the dialogue up to (d), but doesn't discuss (e-f). At (f),   I think, the analysis  fails unless we manage  somehow to block the   default restriction in this case.
 
 
In our proposal, instead, we let  questions introduce topics which then can restrict the belief state. (The relation between intermediate accommodation and topical domain restriction has been studied by David Beaver, but I am a bit confused now about it). The set of topics grows as the discourse proceeds and so,  by extending the topics domain,  we mimic belief revision. 

\ex. 
\a. A or B? \hfill  set up topics: A,B
\b. A \hfill interpreted wrt (A,B)
\c. But, A or C? \hfill add topic:  C
\d. I don't know. \hfill interpreted wrt (A,B,C)
\d. So, A or B? \hfill no  new topics
\d. I don't know, not B. \hfill interpreted wrt (A,B,C)

I think this is a fair account, but we have unfortunately the polar question problem.
\paragraph{The  polar questions problem} 
We have three options as to the topic set up by a polar question ?A:

 

\ex.     
\a. $\{ A \}$ 
\b. $\{A, \neg A\}$
\c. no topic

Option (a)   is what we have in Aloni and van Rooij, but of course we get wrong results once we use it to restrict the agent's belief state. Option (b) is what we had in the appendix
 but, as we saw, gives problems in a dynamic setting because it trivializes subsequent restrictions. Option (c) should work for the restriction purposes, but then we fail to account for the question meaning in terms of topics, because of course all polar questions will have the same topic content, the empty set. But still in what follows a description of how such an account could look like. I have tried to keep it as simple as possible, but it might still be too intricate.
}
\subsection*{New appendix}

 In this appendix we present a dynamic semantics for questions and
their embedding under ``know". The system is one way of
formalizing Schaffer's remarks on the dynamics of conversation,
but making explicit the importance of the order in which questions
are
asked.

\paragraph{Question representation} 

Questions are represented  by   formulas of the form $?  p_1, ..., p_n \ \phi$ where ? is a query-operator, $p_1, ..., p_n$ is a possibly empty sequence of propositional variables, and $\phi$ is a formula of   predicate logic with   propositional variables.   Polar questions result when the query-operator binds no variable  (example \Next-a). Alternative questions are represented by   formulas like \Next-b which asks which of the propositions $\phi$ and $\psi$ is true. Formula \Next-c represents constituent questions.  It can be paraphrased as `which of the propositions   $ \phi(d),   \phi(d'),... $ are true?'.
\ex.
\a. Polar questions: $?  \phi$
\b. Alternative questions: $?p ( p \wedge   (p=\phi \vee p=\psi)    )$
\c. Constituent questions: $?p (p \wedge  \exists x (p=\phi(x)) )$


\paragraph{Question denotation}
Questions are mapped into sets of   pairs   $\langle \sigma,w\rangle$ where $\sigma=\alpha_1,...,\alpha_n$ is a   sequence of propositions and $w$ is a possible world. From such sets of pairs, representing the denotation of question $Q$, we will be able to recover (a) the partition induced by the question  and (b) the topics  set up by $Q$. In terms of partitions we will define various notions of answers. Topics will be used to model the contextual restriction  at work in Schaffer's examples.


Let $\lin \phi\rin_{M,g}$ be the  proposition expressed by an indicative sentence $\phi$:\footnote{The satisfaction relation $\models$, assumed in \ref{prop},  is defined in a standard way. The only clause which deserves some attention is the one for propositional identity:
 
 \ex. $M,w   \models_g  \phi=\psi \ $ iff $ \ \forall w': M,w'   \models_g  \phi $ iff $M,w'   \models_g \psi$ 

}

\ex. \label{prop} $\lin  \phi\rin_{M,g} = \{   w  \mid  \ M,w   \models_{g} \phi \}$. 
 
Questions denotations are then defined as follows, where
 $\vec{p}$ stands for the sequence $p_1,...,p_n$, and  $\vec{\alpha}$ for  the sequence $\alpha_1,...,\alpha_n$.  
\ex. \label{qden} $\lin? \vec{p} \ \phi\rin_{M,g} = \{ \langle \vec{\alpha}, w\rangle \mid  \ w \in  \lin  \phi\rin_{M,g[\vec{p} / \vec{\alpha}  ]}  \}$ 
 
 
  
\noindent The denotation of a  polar question $?  \phi$ is the set of pairs $\langle  \lambda, w \rangle$ such that $\lambda$ is the empty sequence and $w$ satisfies $ \phi$.
 The denotation of an alternative question $?p (p \wedge    (p=\phi \vee p=\psi))$ is the set of pairs $\langle  p, w \rangle$ such that $w$ satisfies $p$ and either $p$ is the proposition expressed by $ \phi $ or it is the proposition expressed by $  \psi $. And finally the denotation of a constituent question represented as  $ ?p ( p \wedge \exists x (p=\phi(x)) )$ is the set of pairs $\langle  p, w \rangle$ such that $w$ satisfies $p$ and $p=\lin \phi\rin_{M,g[x/d]} $ for some individual $d$. 

   
   
\notext{The denotation $\lin ? \phi\rin_{M,g} $ of a  polar question $?  \phi$ is   the set of pairs $\langle  \lambda, w \rangle$ such that $\lambda$ is the empty sequence and $w$ satisfies $  \phi$.
 The denotation of an alternative question $?p (p \wedge    (p=\phi \vee p=\psi))$ is the set of pairs $\langle  p, w \rangle$ such that $w\in p$ and either $p=\lin  \phi\rin_{M,g}$   or   $p=\lin  \psi\rin_{M,g} $. And finally the denotation of a constituent question  $ ?p ( p \wedge \exists x (p=\phi(x)) )$ is the set of pairs $\langle  p, w \rangle$ such that $w\in p$ and $p=\lin \phi\rin_{M,g[x/d]} $ for some individual $d$. }
 
 \notext{Definition \ref{qden} assigns then the following denotations to  polar, alternative and constituent questions:
\ex.  
\a.   $\lin ?\phi\rin_{M,g}=\{\langle \lambda, w \rangle \mid  w \in \lin \phi \rin_{M,g}  \}$
\b.    $\lin?p (p \wedge    (p=\phi \vee p=\psi))\rin_{M,g}  =
 \{\langle    p , w \rangle \mid p =\lin \phi\rin_{M, g} $ or $p= \lin \psi\rin_{M,g} \ \& \  w \in p \}$
\c.   $\lin?p ( p \wedge \exists x (p=\phi(x)) )\rin_{M,g} = \{ \langle p, w\rangle \mid \exists d: p=  \lin \phi\rin_{M,g[x/d]} \ \& \ w \in p \}$

%\b.    $\lin?p (p \wedge    (p=\phi \vee p=\psi))\rin_{M,g}  =
 %\{\langle    \lin \phi\rin_{M,g} , w \rangle \mid w \in \lin \phi\rin_{M, g} \} \cup  %\{\langle    \lin \psi\rin_{M,g}, w \rangle \mid w \in \lin \psi\rin_{M, g} \}$ 
  
 
  

} 
\paragraph{Answers} The denotation   of a question  $Q$ determines
an equivalence relation (or, equivalently, a partition) Part($Q$)
over the set of possible worlds, from which different notions of
answers can be defined. For $Q=?\vec{p}  \ \phi$, two worlds $w$ and $v$ are in the same
cell of Part($Q$) if for each sequence of propositions $\vec{\alpha}$, $\langle \vec{\alpha}, w\rangle $   belongs to    the denotation of $Q$  iff $\langle \vec{\alpha}, v\rangle $ does as well. 


 

 

\ex. Part$_{M,g}(?\vec{p}  \ \phi)$ =$\{\langle  w,v\rangle \mid  \forall \vec{\alpha}:      \langle \vec{\alpha}, w \rangle \in  \lin?\vec{p} \ \phi\rin_{M,g} $ iff  $ \langle \vec{\alpha} , v \rangle \in  \lin ?\vec{p} \ \phi\rin_{M,g} \}$  

 \notext{ True exhaustive answer wrt $w$:

\ex. TEA$_w (?p_1,...,p_n  \phi)$=$\{ w'  \mid  \forall \alpha_1,...,\alpha_n:      \langle \alpha_1,...,\alpha_n, w \rangle \in  [?p_1,...,p_n  \phi] $ iff  $ \langle \alpha_1,...,\alpha_n, w' \rangle \in  [?p_1,...,p_n  \phi] \}$ }

\noindent \emph{Exhaustive answers} to $Q$ correspond to cells in
Part($Q$). The \emph{exhaustive true answer} to $Q$ in $w$ is the
cell including $w$ in Part($Q$). Finally, \emph{partial answers}
(true in $w$) correspond to non-trivial unions of cells of
Part($Q$) (including $w$), namely unions of cells different from
Part($Q$).


\paragraph{Topics} In terms of a question denotation we can also define the topics set up by the question as follows:

\ex. Top$_{M,g}(?\vec{p} \ \phi)$ = $\{  \vec{\alpha}  \mid \exists w:  
  \langle  \vec{\alpha} , w\rangle  \in \lin ?\vec{p}  \ \phi\rin{m,g}  \}$

Topics are sets of (sequences of) propositions. These topics will be used in what follows to define the contextual restriction on the belief state of the subject of   knowledge-wh.
\paragraph{Results:}

\ex. Polar questions (``did John leave?")
\a. Representation: $?\phi$
\b. Partition: $\{\phi, \neg \phi\}$
\b. Topics: $\{ \lambda\}$ (the set containing the empty sequence)


\ex. Alternative questions (``did John leave, or did Mary leave?")
\a. Representation: $?p ( p \wedge (p=\phi \vee p=\psi))$
\b. Partition: $\{\phi \wedge \neg \psi, \neg \phi \wedge   \psi, \neg (\phi\vee \psi), \phi \wedge \psi\}$
\b. Topics: $\{\phi,\psi\}$


\ex. Constituent questions (``who left?")
\a. Representation: $?p( p \wedge \exists x (p=\phi(x)))$
\b. Partition: $\{\forall x  \neg \phi(x), \forall x (\phi \leftrightarrow x=d), ..., \forall x \phi(x)\}$
\b. Topics: $\{\phi(d),\phi(d'), ...\}$


\paragraph{Dynamics}


A context $C$ is defined as an ordered pair whose first index
$s_C$ is an information state (set of worlds), and whose second
index $i_C$ is a sequence of question denotations (a sequence of sets of   sequence-world pairs)  representing  the issues  under discussion in $C$. A context $C$ can be updated either by an assertion
$P$, or by the introduction of a new question $Q$:

\ex. \a. $C+P = (s_C \cap \lin P\rin, i_C)$
\b. $C+Q= (s_C, i_C+\lin Q\rin)$

Updating $C$ with an assertion $P$ only influence the state-index    eliminating those worlds in $s_C$ in which $P$ is false.  Updating with a question $Q$ extends the issue parameter by adding to $i_C$ the denotation of $Q$ as last issue under discussion.
 
\paragraph{Knowledge} Let ANS$_w(Q)$ be the true exhaustive answer to $Q$ in $w$, and Top$(C)$ denote the union of the topics introduced by  all  the issues in $C$, i.e.  for $C= (s_c,\lin Q_1 \rin, ..., \lin Q_n \rin)$:
Top$(C)$ = $\bigcup_{i\in n} $Top$(Q_i) \setminus \{\langle \rangle\}$.
We then define knowledge as follows:

\ex. ``S knows Q'' is true in world $w$ with respect to context $C$ iff \a. $K_w \ \cap\ $Top$(C) \ \subseteq \ $ANS$_w(Q)$, if  Top$(C)\neq\emptyset$; \b. $K_w   \ \subseteq \ $ANS$_w(Q)$, otherwise.


  


 


\end{document}

 

 \paragraph{know versus forget}
 
 If we assume, following Higginbotham, that {\it whether} is a universal quantifier,
 we get a clear difference between the case of {\it know} and the case of {\it forget}. 
 Analysis, roughly, (we could get this in an elegant way in a Hamblin semantics)
 \ex.  
 \a. John knows whether A or B.
 \b. $\forall p [p=A \vee p=B]:$ John knows whether $p$
 
 \ex. \a. John forgot whether A or B
 \b. $\forall p [p=A \vee p=B]:$ John forgot whether $p$
 
I think this analysis fully explains Schaffer's intuition in the case of {\it forget}. To forget whether A or B you need to forget whether A and to forget whether B. We then correctly predict that we can have a difference   in truth conditions in the following examples, in a situation in which it holds that John knows that his keys are not in the fridge. 

\ex.  \a. John forgot whether he left the keys on the table or in the bag. (TRUE)
\b. John forgot whether he left the keys on the table or in the fridge.  (FALSE)

The case of {\it know} is more complex, in agreement with my intuitions. In the situation described by Schaffer (in which it is Bush on TV) we still have that if (a) is true then (b) is also true. 
\ex. \a. John knows whether it is Bush or Jackson.   
\b. John knows whether it is Bush or Ferrell.

To explain his intuitions we need pragmatics.

To conclude: Our intuitions in the case of {\it forget} are sharp, in the case of {\it know} are more tricky. Schaffer, wrongly,  does not distinguish between the two cases. We, instead,  can distinguish them, if we assume this analysis. The case of {\it forget} is explained by pure semantics, the case of {\it know} requires `smart' pragmatics. 

  
\end{document}

