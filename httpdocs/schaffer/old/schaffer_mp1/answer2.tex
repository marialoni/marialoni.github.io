\section{Knowing the Answer}

The analysis sketched in the previous section and developed in the
Appendix is one way of formalizing Schaffer's contextualist
intuition. We therefore agree with Schaffer that the truth-value
of knowing-\emph{wh} ascriptions will vary depending on the
background of possibilities that are considered relevant. However,
we remain skeptical on the idea that this should provide an
argument against the so-called reductive analysis of
knowing-\emph{wh}, or even against the soundness of the schema of
convergent knowledge.

As regards the latter, the reason is that if the background
context $X$ is kept constant in $(19)$, then convergent questions
still yield equivalent knowledge claims, and as explained in
general $X$ is not simply a function of the question asked. For
instance, relative to $X=B\cup F\cup J$, not only does S fail to
know whether Bush or Ferrell is on TV, but we have seen that S
does not really know whether Bush or Jackson is on TV either.
Conversely, if one can indeed say ``S knows whether it is Bush or
Janet Jackson" and then ``but S does not know whether it is Bush
or Ferrell", then this is due precisely to the fact that the
background context $X$ differs in the two cases.

As regards the reductive view of knowing-\emph{wh}, Schaffer
denies it by writing: ``there is more to knowledge-\emph{wh} than
knowing the proposition that just so happens to be the answer". We
do not quite agree with this view. For one thing, even a
context-dependent analysis of knowing-\emph{wh} like \ref{cd} is
expressed in terms of knowing-\emph{that}. More importantly, we
take Schaffer's example to be revealing rather of a form of
ambiguity in what ``knowing the answer" means in the first place.
There may be two senses of ``knowing the answer to the question
whether A or B": ``knowing whether A or B" \emph{before} the
question is asked explicitly by someone else, and ``knowing
whether A or B" \emph{after} the question has been asked
explicitly by someone else. A related point was made earlier by J.
Hawthorne, who writes that ``the very asking of a question may
provide one with new evidence regarding the subject
matter".\footnote{Hawthorne (2004: 78), quoted by Schaffer (2005),
fn. 22.} However, Schaffer dismisses it by considering that we can
make knowing-wh ascriptions \emph{in absentia}, namely ``of
subjects who are miles away". We agree with him, but in our
opinion, knowledge ascriptions may differ depending on whether the
ascriber imagines: ``if I asked him whether A or B, how would he
answer?", or: ``what does he presently think about whether A or
B?".

Consider a pupil, on the way to her history exam, trying to
remember whether Napoleon I was born in 1768 or 1769. She cannot
remember whether Napoleon was born in 1768 or 1769. Then comes the
exam; one of the questions is: ``was Napoleon born in 1769, or in
1869?". S answers that Napoleon was born in 1769, because she
knows for sure that Napoleon was born before 1800. By doing so,
however, S makes the reasonable assumption that the question asked
contains the correct answer. In other words, S accommodates the
information contained in the question in order to restrict the
range of possibilities. Before the question was asked, however,
can one say:

\ex. S knows whether Napoleon was born in 1769 or in 1868

This is not clear. For once again, S's knowledge that Napoleon was
not born in 1869 is not sufficient to settle the issue. After the
question was asked and S correctly answers ``1769", however, it
seems one can indeed say:

\ex. S knows the answer to the question whether Napoleon was born
in 1769 or 1869.

\noindent If so, the contrast suggests that ``knowing whether" and
``knowing the answer to the question whether" are not necessarily
synonymous, or more precisely, that ``knowing whether", just like
``knowing the answer", is potentially ambiguous.

The example points toward another interesting phenomenon. Before
the question was asked, it seems also infelicitous to say:

\ex. S does not know whether Napoleon was born in 1769 or in 1869

For in this context, ``S does not know whether A or B" seems to
presuppose ``S wonders whether A or B".\footnote{Remember that
this is before anyone other than S asks the question. In this
context, ``wonder" is stronger than ``being asked", since it means
``asking oneself".} But wondering whether A or B implies that one
considers both A and B as possible. In our scenario, S wonders
whether Napoleon was born in 1768 or 1769, but does not wonder
whether he was born in 1869, because she already rules out that
possibility.
%In that case, it is simply odd to say: ``S knows whether Napoleon
%was born in 1768 or 1868", because S does not entertain the second
%possibility.
\ This presuppositional behavior accounts for others of Schaffer's
examples. Thus, Schaffer notes that the two sentences:

\ex. \a. \label{phone} I forgot whether I left my keys on the
table, or by the phone. \b. \label{fridge} I forgot whether I left
my keys on the table, or in the fridge.

\noindent are clearly inequivalent in a context in which I
remember I did not leave my keys in the fridge, but don't remember
whether I left them on the table or by the phone, even if the two
embedded questions have the same answer (I left my keys on the
table). According to Schaffer, \ref{phone} is true and
\ref{fridge} is \emph{false} in that situation. On our view,
\ref{fridge} is \emph{odd} or \emph{inappropriate} rather than
false, since in order for \ref{fridge} to receive a truth value,
the presupposition that I wonder whether I left my keys in the
fridge ought to be satisfied. We perfectly agree, therefore, about
the inequivalence, but disagree on the nature of the
inequivalence. On our account, the intuition of inequivalence is
easily explained if one adopts GS's view of \emph{wonder} as
selecting the intension of a question: thus, the two questions
``whether I left my keys on the table or by the phone" and
``whether I left my keys on the table or in the fridge" clearly
have distinct intensions. Those, however, affect the
presuppositional content of \emph{forget whether}, rather than its
assertive content.

More generally, the understanding of embedded questions will
differ depending on the embedding verb and its presuppositional
behavior. There is, in this respect, a clear contrast between
``know whether" and ``can tell whether", or more clearly, ``could
tell whether". Let us compare for instance:

\ex. \a.\label{know} S knows whether Napoleon was born in 1769 or
in 1869. \b.\label{tell} S could tell whether Napoleon was born in
1769 or in 1869.

Our intuition is that \ref{tell} is more appropriate than
\ref{know} for the kind of scenario that Schaffer considers, since
it refers more clearly to the fact that S is facing or might be
facing an explicit question. More precisely, ``S could tell
whether A or B" seems to mean: ``supposing A or B were actually
the case, S could tell which of them is", thereby justifying the
restriction of the alternatives to A and B.\footnote{Thus, in the
Bush-Ferrell scenario, for instance, it seems one can even say:
``S could tell whether it is Ferrell or Janet Jackson", meaning
that if it had to be only one of them, S could still rule out
Janet Jackson and choose Ferrell.} We do agree, however, that
``know" can be used to mean ``can tell", and likewise ``would
know" can be used to mean ``could tell". Nevertheless, this sense
of ``know" is compatible with weaker standards of knowledge. To be
sure, let us compare:


\ex.\a. (?) S does not know for sure whether Napoleon was born in
1768 or in 1769, but he knows whether Napoleon was born in 1769 or
in 1869. \b. S does not know for sure whether Napoleon was born in
1768 or in 1769, but he could tell whether Napoleon was born in
1769 or in 1869.

%\a. (??) S knows whether Napoleon was born in 1768 or in 1868,
%although he does not know for sure whether Napoleon was born in
%1768 or 1769. \b. S can tell whether Napoleon was born in 1768 or
%in 1868, although he does not know for sure whether Napoleon was
%born in 1768 or 1769.

%

Pierre can tell whether A or B if he knows and can provide the
partial answer ``not A", even if he does not know about B for
sure. Thus, ``can tell whether A or B" seems more adequate than
``know whether" to mean ``make the correct choice between A and
B". By contrast, for Pierre to know whether A or B, Pierre may
well have to know whether A \emph{and} to know whether B. Knowing
whether, we argue, is ambiguous between these two senses of
knowing the answer: knowing a partial answer to the question, and
being able to infer the actual answer from that partial answer on
the one hand, versus knowing the true exhaustive answer directly
on the other hand, independently of the information brought up by
the questioner. The latter, of course, is more demanding than the
former.\footnote{In some cases, ``knowing the answer" may simply
mean ``being able to give the correct answer", whatever method has
been used, even if S has simply \emph{guessed} the correct answer.
However, ``S could tell whether A or B" seems to imply that S has
a reliable method to tell the correct answer.} As Schaffer writes,
``a student might well know the answer when the options are easy".
This, however, is ``knowledge in the eye of the ascriber". Suppose
the student answers ``1769" to the question whether Napoleon was
born in 1769 or 1869. By the professor's standards, the student
knows the answer. But does S know that she knows the answer? This
seems less obvious, because S may still doubt whether Napoleon was
born in 1769, and fear the professor might have devised a ``snare
question". For S to know that she knows the answer, S would have
to know that Napoleon was born in 1769 prior to the question being
asked, without having to trust the questioner on the correctness
of the alternatives.
