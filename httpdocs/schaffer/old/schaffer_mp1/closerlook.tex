\section{A closer look at Schaffer's examples}

%A feature of alternative questions is that they presuppose that
%only one of the disjuncts is true.

By asking an alternative question, one generally presupposes that
one and at most one of the two disjuncts is true.\footnote{It is
controversial, however, whether this presupposition should be part
of the denotation of alternative questions. We maintain, in the
semantics presented in the Appendix, that the answer to an
alternative question of the form ``whether A or B" could be
``both" or ``neither".} \ For instance, when asking
\ref{jackson}-a, one typically presupposes that exactly one of
Bush or Jackson is on television. This, however, does not imply
that there could not be more than one person on television. In a
context in which Bush is on TV and Janet Jackson is not, the true
exhaustive answer ``Bush is on TV and Janet Jackson is not on TV"
to \ref{jackson}-a does not exclude that Ferrell or someone else
too might be on TV (along with Bush). For the kind of scenario
that Schaffer describes, however, the participants to the
conversation presumably share the extra knowledge that
\emph{exactly one person is on TV}. If we add this pragmatic
presupposition into the picture, then the two questions
\ref{jackson}-a and \ref{ferrell}-b, on their alternative reading,
will indeed converge to the same proposition:

\ex. \label{noone} Bush and noone else is on television.

%

Interestingly, an answer of that kind counts as a strongly
exhaustive answer to the question ``who is on television?". Thus,
if you know that Bush and noone else is on television, then you
know who is on TV. In Groenendijk and Stokhof's framework, the
question ``who is on TV?" determines a partition of the set of
possible worlds, where each cell in the partition represents a
strongly exhaustive answer to the question. Assuming that the
domain of individuals consists of only George W. Bush, Will
Ferrell, and Janet Jackson, for instance, then the question ``who
is on TV?" would be represented as in Figure 1, where $B^{*}$
denotes the set of worlds in which Bush and no one else is on TV,
$J^{*}$ the set of worlds in which Jackson and no one else is on
TV, and $F^{*}$ the set of worlds in which Ferrell and no one else
is on TV (and likewise $(BF)^{*}$ the set of worlds where Bush and
Ferrell and no one else is on TV, and so on).\footnote{The symbol
$^*$ is used to mark the ``and no one else". We abbreviate
$\val{P}$ into $P$: e.g. $B$ denotes the proposition that Bush
(and possibly someone else) is on TV, and $B^{*}$ the proposition
that Bush and no one else is on TV (ie $B \overline{F}\
\overline{J}$).}

\begin{figure}[t]


\centering{


\begin{tabular}{|c|}
\hline $\emptyset$\\ \hline
$B^{*}$\\
\hline
$F^{*}$\\
\hline
$J^{*}$\\
\hline
$(BF)^{*}$\\
\hline
$(BJ)^{*}$\\
\hline
$(FJ)^{*}$\\
\hline
$(BFJ)^{*}$\\
\hline
\end{tabular}}

\setlength{\unitlength}{0.5cm}

\begin{picture}(5,3)(1.7,-8)

\put(5,1){\oval(0.6,1.6)}

\put(4.9,1.3){\tiny{$w$}}

\put(5.2,0.8){\line(1,1){1}}

\put(6.3,1.8){$K_w$}

\end{picture}

\vspace{-1.3cm}

\caption{Who is on TV?}

\end{figure}


By hypothesis, the actual world $w$ is contained in $B^{*}$. If it
is part of the common ground that exactly one person is on TV,
then all possibilities outside $B^{*}$, $F^{*}$ and $J^{*}$ are
excluded when the dialogue starts. Schaffer does not specify in
which order the questions are asked. This matters of course, since
questions, like assertions, can change the context. In particular,
they can change the range of alternatives the attitude holder is
aware of, and likewise the range of alternatives considered
relevant by the ascriber. For simplicity, we consider the
situation statically first, and assume that the total state of
knowledge of the agent can be represented as in Figure 1, taking
into account the three possibilities all at once.


%and
%assume that the three answers $B^{*}$, $J^{*}$ and $F^{*}$ are
%equally relevant.
\ Let $K_w$ denote the set of worlds compatible with the beliefs
of the agent (call him S) who is asked the question at $w$. We
assume the standard truth conditions for knowing-\emph{that},
namely:

\ex. ``S knows that P" is true in $w$ iff $K_w \subseteq P$

One way of representing the scenario Schaffer has in mind is to
suppose that $K_{w} \subseteq B^{*}\cup F^{*}$, but $K_{w}
\nsubseteq B^{*}$ and $K_{w}\nsubseteq F^{*}$. That is, S thinks
that one of Bush or Ferrell is on TV, but doesn't know which.
Consequently, S knows that the person on TV is not Janet Jackson,
since her belief set excludes any Janet world.

It is clear in what sense S does not know whether Bush or Ferrell
is on television, since S's belief set entails neither $B^{*}$ nor
$F^{*}$, but overlaps both. But in what sense can it be said that
S knows whether it is Bush or Janet Jackson? Schaffer's idea here
is that if S were to choose between the two incompatible answers
``Bush is on TV", and ``Janet Jackson is on TV", then S would
choose the first. In other words, if S can ignore the possibility
that it might be Ferrell, then her belief state would be contained
entirely in $B^{*}$. On a first approximation, this suggests the
following truth-conditions for knowing-\emph{whether} with
alternative questions:

\ex. \label{wh} ``S knows whether A or B" is true in $w$ iff
$K_{w}\cap(A \cup B) \subseteq A$ or $K_{w}\cap(A \cup B)
\subseteq B$

These truth-conditions are faithful to Schaffer's contextualist
idea. In particular, \ref{wh} entails that S knows whether it is
Bush or Janet Jackson who is on TV, since $K_{w}\cap (B \cup
J)\subseteq B$. But S does not know whether it is Bush or Ferrell
who is on TV, since $K_{w}\cap(B \cup F)\nsubseteq B$ and
$K_{w}\cap(B \cup F)\nsubseteq F$. This analysis can be made
compositional if we suppose, as Schaffer does, that for an
alternative question Q of the form ``whether A or B", the meaning
of Q is defined directly as the set of potential answers $\{A,
B\}$.\footnote{See Schaffer 2005, fn. 9. This way of defining the
meaning of a question, as the set of possible answers, is closer
to Hamblin's 1973 than to either Karttunen's or Groenendijk and
Stokhof's.} Let \val{Q}($w$) denote which of those alternatives is
true in $w$. Then \ref{wh} can be rephrased as:

\ex. \label{cd} ``S knows Q" is true in $w$ iff $K_{w}\cap
(\cup\val{Q}) \subseteq \val{Q}(w)$

%
By contrast, the truth-conditions given initially in \ref{ws} for
knowing-\emph{wh} were of the form:

\ex. \label{cf} ``S knows Q" is true in $w$ iff $K_{w} \subseteq
\val{Q}(w)$

%

\noindent The difference is that in \ref{cd} the ascription of
knowledge is context-dependent, and more precisely, it is
dependent on the question itself. Moreover, the truth-conditions
of \ref{cd} no longer validate (CV), since precisely from
\val{Q}($w$)=\val{Q'}($w'$), it needn't follow that
$\cup\val{Q}=\cup\val{Q'}$.

The general pattern underlying \ref{cd} is of the form:

\ex. \label{gen} ``S knows Q" is true in $w$ with respect to $X$
iff $K_w \cap X \subseteq \val{Q}(w)$

In agreement with Schaffer, the parameter $X$ represents a
contextual restriction of the background of possibilities relevant
for the answer. In \ref{cd} this restriction is derived directly
from the semantics of the alternative question (and therefore the
restriction does not appear on the left hand side of the
equivalence). This may be too drastic, however, since this
analysis predicts that the order in which the questions are asked
will not matter. Suppose S is asked first whether it is Bush or
Janet Jackson. Then there are two possibilities. Either S is not
yet entertaining the possibility that it might be Ferrell, and
blindly answers: ``Bush". Or S is aware of the possibility that it
might be Ferrell, but understands that the correct answer is
either ``Bush" or ``Janet Jackson". Either way, the ascriber, who
knows the correct answer, can say: ``S knows whether it is Bush or
Janet Jackson".

But let us imagine that S is asked first whether it is Bush or
Ferrell. This time, S answers: ``I don't know". Then S is asked
whether it is Bush or Janet Jackson. It is not obvious that S will
answer: ``it is Bush", if S still entertains the possibility that
it might be Ferrell. In other words, assuming $X=B\cup F\cup J$
when the second question is asked (namely that the three
alternatives are relevant), then it is no longer true that $K_w
\cap X \subseteq B$. In that situation, the best answer S can give
is: ``at any rate, it is not Janet Jackson", since $K_w \cap
(B\cup F\cup J)$ still excludes $J$. But that answer may not be
sufficient to say that S knows whether it is Bush or Janet
Jackson. With respect to the question: ``who is on TV?", the
answer ``it is not Janet Jackson" is what Groenendijk and Stokhof
call a \emph{partial answer}, since it does not select one cell in
the partition induced by the question. But knowing that Janet
Jackson is not on TV is not sufficient in general to know who is
on TV.

%More generally, the analysis in \ref{cd} loses the entailment from
%``S knows whether Bush or Janet Jackson is on TV" to ``S knows who
%is on TV". For \ref{cd} predicts that S knows whether it is Bush
%or Jackson, and does not know whether it is Bush or Ferrell, but
%from the latter it should follow that S does not know who is on
%TV.

%More generally,
%without corresponding modifications on the analysis of
%who-questions,
%the analysis in \ref{cd} also loses the entailment from ``S knows
%whether Bush or Janet Jackson is on TV" to ``S knows who is on
%TV", unless

For an analysis like \ref{cd} to work, one would have to suppose
that the subject S, when asked an alternative question of the form
``is it A or B?", systematically accommodates the presupposition
that one and only one of the two alternatives is true.\footnote{A
similar objection was made to Schaffer by J. Stanley and another
participant at the Aberdeen Conference on Linguistics and
Epistemology, 2007.} But this may not always be the case. Suppose,
to use Dretske's notorious example, that a skeptic first asks:
``is it a zebra or a horse?". I answer: ``a zebra". He pursues:
``but is it a zebra, or a cleverly painted mule?". I answer: ``I
don't know". Then the skeptic reiterates: ``so is it a zebra, or a
horse?". Here, the best answer can only be: ``not a horse"
(assuming I remain confident about this), for in that state the
skeptic undercuts the invitation to consider that exactly one of
the two answers is true. This suggests that \ref{cd} should be
revised in a way that integrates the dynamics of questions and the
order in which they introduce alternatives.

In the Appendix, we present a dynamic system with
context-sensitive truth conditions for knowledge attributions
along the lines of \ref{gen}, which is aimed to capture precisely
this phenomenon. The truth-conditions we obtain differ from
\ref{cd} in three respects. First, the contextual restriction for
knowing-\emph{wh} is not simply a function of the embedded
question; rather, it is a function of the question under
discussion, in agreement with Schaffer's own account, namely of
the alternatives judged relevant by the ascriber, even though the
embedded question remains an input to the question under
discussion. Secondly, this contextual restriction is defined in an
incremental way, relative to the previous questions under
discussion in the conversation, in order to model the dynamics of
conversation. Finally, the meaning of embedded questions remains
defined in terms of the partition induced by the question, but
partitions are derived from a Hamblin style definition of the
denotation of questions. The system allows us to implement
Schaffer's contextualist idea, but in a sense we predict more
context-sensitivity than Schaffer suggests. In particular,
Schaffer seems to consider that even if the alternatives consist
of Bush, Ferrell and Janet Jackson, ``virtually anyone (with
decent vision and minimal cultural background) can know whether it
is Bush or Janet Jackson". In our approach, assuming that S's
belief set is as in Figure 1, this is so only if the alternatives
are restricted to Bush and Jackson, or otherwise if
knowing-\emph{wh} is given the weak sense of knowing a partial
answer to the question.


%A difference with Schaffer's informal approach is that Schaffer
%writes as if knowing whether it is Bush or Janet Jackson was
%always easy, as opposed to knowing whether it is Bush or Ferrell.
%But...
