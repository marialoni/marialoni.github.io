\section{Conclusion}

Schaffer's examples make a new case for the context-sensitivity of
knowledge ascriptions. While we agree on the contextualist
conclusion Schaffer draws from his examples, we nevertheless
disagree with him on several points of details, and more
substantially, on the extent given to the contextualist analysis.
In the first section of this paper, we have argued that the schema
of convergent knowledge does not provide a straightforward
argument against the reductive analysis of knowledge-\emph{wh} to
knowledge-\emph{that}, in particular if one adopts the partition
theory of questions. In the last part, we have moreover argued
that some of the inequivalences observed by Schaffer should be
traced to the presuppositions associated with the use of
\emph{whether}-complements, rather than to the assertive content
they contribute under the scope of attitude verbs. More
fundamentally, on our account Schaffer's examples are revealing of
a form of ambiguity in what ``knowing the answer" means, roughly
between knowing the true exhaustive answer prior the question is
asked, and being able to tell the answer if the question were
asked explicitly (relying on knowledge of the partial answer).
Like Schaffer, however, we do agree that the meaning of questions
is dependent both on the domain of quantification and on the
alternatives that are considered relevant. But this is not
sufficient to conclude, as Schaffer does, that
knowledge-\emph{that}, no more than knowledge-\emph{wh},
systematically ``includes a question". On the present view,
``know" continues to denote a binary relation between an agent and
a proposition, even when the context is more richly articulated.
