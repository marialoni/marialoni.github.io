\documentclass[font=palatino]{beamer}



%\usecolortheme[rgb={0.6,0.2,0.2}]{structure}
\usepackage{beamerthemeMalmoe}

\mode<presentation> \AtBeginSubsection


\title{Alternative Questions\\ and Knowledge Attributions}


\date{18/04/2008}
\author[]{M. Aloni \& P. \'Egr�}
\institute{}

\AtBeginSubsection[] {
  \begin{frame}<beamer>
    \frametitle{Plan}

    \tableofcontents[currentsection,currentsubsection]
  \end{frame}
}


\newcommand{\lang}[1]{\ensuremath{\mathcal{L #1}}}

\usepackage{latexsym}
\usepackage[cyr]{aeguill}

\usepackage{amsthm}
\usepackage{amsmath}
\usepackage{amssymb}
\usepackage{times}
\usepackage{bm}
\usepackage{linguex}
\newtheorem{theo1}{Theorem}
\newtheorem{conjecture}{Conjecture}
\newtheorem{def1}{Definition}
\newtheorem{thez}{Thesis}
\newtheorem{question}{Question}
\newtheorem{loi}{Law}
\newtheorem{hypot}{Hypothesis}
\newtheorem{lemme}{Lemma}

\newcommand{\bl}{$\bullet$\ }
\newcommand{\val}[1]{\mbox{ $[\![$  #1 $]\!]$}}
\newcommand{\bit}{\begin{itemize}} \newcommand{\eit}{\end{itemize}}
\newcommand{\ben}{\begin{enumerate}} \newcommand{\een}{\end{enumerate}}
\newcommand\notext[1]{}


\newcommand{\la}{\langle}
\newcommand{\ra}{\rangle}
\newcommand{\LA}{\langle}
\newcommand{\RA}{\rangle}
\newcommand{\linf}{$L_{\infty, \infty}$}
\newcommand{\linfm}{$L^{-}_{\infty, \infty}$}
\newcommand{\lio}{$L_{\infty, \omega}$}
\newcommand{\lom}{$L_{\omega, \omega}$}
\newcommand{\struc}[1]{\ensuremath{\la #1 \ra}}
\newcommand{\iso}[1]{\ensuremath{\approx _{ #1 }}}
\newcommand{\isoa}{\ensuremath{\approx _{\alpha}}}
\newcommand{\isop}{\ensuremath{\approx _{p}}}
\newcommand{\nisop}{\ensuremath{\not \approx _{p}}}
\newcommand{\sta}{\ensuremath{\mathcal{A}}}
\newcommand{\stb}{\ensuremath{\mathcal{B}}}
\newcommand{\stc}{\ensuremath{\mathcal{C}}}
\newcommand{\stm}{\ensuremath{\mathcal{M}}}
\newcommand{\stmp}{\ensuremath{\mathcal{M}'}}
\newcommand{\eqliw}{\ensuremath{\equiv _{\infty , \omega} }}
\newcommand{\eqliwa}{\ensuremath{\equiv _{\infty , \omega}^{\alpha} }}

%%%%%%%%%%%%%%%%%%%%%%%%%%%%%%%%%%%%

\begin{document}



\begin{frame}

\maketitle

\end{frame}

%%%%%%%%%%%%%%%%%%%%%%%%%%%%%

\section{Convergent Knowledge}

\begin{frame}

\frametitle{Schaffer's problem of convergent knowledge}

\framesubtitle{(J. Schaffer 2007)}

\ex. \a. Is George Bush or Will Ferrell on TV? \b. Is George Bush
or Janet Jackson on TV?\pause

\begin{itemize}

\item \alert{Assumption 1}: to know Q is to know the true answer
to Q

\item \alert{Assumption 2}: ``George Bush is on TV" is the true
answer to both questions.



\end{itemize}\pause

\ex. \a. Bill knows whether George Bush or Will Ferrell is on TV.
\b. Bill knows whether George Bush or Janet Jackson is on TV



%%


\end{frame}

\begin{frame}

\frametitle{Convergent knowledge (1)}

\ex.[(CV)] \qquad \begin{tabular}{c}
\val{Q}($w$)\ =\val{Q'}($w$)\\
\hline \val{S knows Q}($w$)\ =\val{S knows Q'}($w$)\\
\end{tabular}\pause

\begin{itemize}

\item (CV) holds both in Karttunen's theory of questions, and in
Groenendijk and Stokhof's theory of questions.\medskip \pause

\item When Q and Q' have the same complete answers, knowing Q and
knowing Q' are truth-conditionally equivalent (by
compositionality).

\end{itemize}

\end{frame}

%%%%%%%%%%%%%%%%%%%%%

\begin{frame}

\frametitle{Convergent Knowledge (2)}

According to Schaffer, ``if a question has a true answer, it must
converge with any question that merely shifts the false answers":

\ex.[(A)] \a. S knows $?(\phi \vee \psi_1)$
 \b.  $\phi$
 \c. S knows whether $?(\phi \vee \psi_2)$

\ex. \a. Bill knows whether Bush or Janet Jackson is on TV. \b.
Bush is on TV. \c. Bill knows whether Bush or Ferrell is on TV.


\end{frame}

\begin{frame}

\frametitle{Schaffer's proposal}

\begin{itemize}

\item the standard analysis of knowing-wh in terms of knowing-that
is flawed.

\item know is question-relative: to know that Bush is on TV means
to know that Bush rather than someone else is on TV

\end{itemize}

\end{frame}

%%%%%%%%%%%%%%%%%%%

\begin{frame}

\frametitle{Our proposal}

\begin{itemize}

\item While (CV) is supported by the standard theories of
questions, (A) need not be.\pause

\item On at least one reasonable analysis of alternative
questions, (CV) fails to deliver Schaffer's predictions\pause

\item Alternative questions can be used to contextually restrict the space of
possibilities relevant for knowledge attributions


\end{itemize}

\end{frame}

%%%%%%%%%%%%%%%%%

\begin{frame}

\frametitle{Outline}

\begin{enumerate}

\item Polar and alternative questions

\item Ascriptions and Context

\item Adding Presupposition

\end{enumerate}


\end{frame}

%%%%%%%%%%%%%%%%%%%%%%%%%%%%%%%%

\section{Alternative Questions}

\begin{frame}

\frametitle{Polar vs Alternative Questions}

\ex. Is Mary French or Italian?\pause

\ex. \a. Is Mary either French or Italian? \b. Yes/No. \c.
*French/*Italian.\pause

\ex. \a. Is Mary [French]$_F$ or [Italian]$_F$? \b. *Yes/*No \c.
French / Italian.

(Cornulier 1982, Haspelmath 2000, Han \& Romero 2003).

\begin{itemize}\pause

\item Schaffer's target: alternative readings

\end{itemize}


\end{frame}

%%%%%%%%%%%

\begin{frame}

\frametitle{Strongly and weakly exhaustive answers}

\ex. Who called?

Context: only Mary and John called.

\begin{itemize}

\item \alert{Karttunen} (weakly exhaustive answer): Mary and John called.

\item \alert{Groenendijk and Stokhof} (strongly exhaustive answer): Mary and
John called, and nobody else called.

\end{itemize}

\end{frame}

%%%%%%%%%%%%%%%%%%%%%%%%%%%%%%%%%

\begin{frame}

\frametitle{Different predictions}

\ex. \a. John knows who called \b. Mary called. \c. John knows
Mary called. [K, GS]\pause

\ex. \a. John knows who called. \b. Sue did not call. \c. John
knows Sue did not call. [GS]

\begin{itemize}\pause

\item In favour of GS: suppose only Mary called. John
knows Mary called, but also believes that Sue called. Whould we
say that John knows who called? (Spector 2006).


\end{itemize}

\end{frame}

%%%%%%%%%%%%%%%%%%%%

\begin{frame}

\frametitle{Partition Semantics}

\ex. Is BUSH or JANET JACKSON on TV?

\begin{itemize}

\item Answers (GS): $\{BJ, B\neg J, \neg B J, \neg B \neg J\}$

\item Assumption (AE): the presupposition, if any, that exactly one of the
disjuncts should be true is not part of the answerhood conditions.

\end{itemize}


\begin{figure}[h]

\setlength{\unitlength}{1.7cm}

\begin{picture}(3,2)

\put(2,2){\line(1,0){2}}

\put(2,1){\line(1,0){2}}

\put(2,0){\line(1,0){2}}

%%%%%%%

\put(2,0){\line(0,1){2}}

\put(3,0){\line(0,1){2}}

\put(4,0){\line(0,1){2}}

%%%%%%

%\put(2,0){\line(1,1){2}}

%\put(2,1){\line(1,1){1}}

%\put(3,0){\line(1,1){1}}

%%%%%%%

\put(1.7,1.5){$J$}

\put(1.5,0.5){$\neg J$}

\put(2.5,2.2){$B$}

\put(3.3,2.2){$\neg B$}

%\put(2.7,1.2){$F$}

%\put(3.7,1.2){$F$}

%\put(2.7,0.2){$F$}

%\put(3.7,0.2){$F$}

\end{picture}

\end{figure}


\end{frame}

%%%%%%%%%%%%%%%%%%%%

\begin{frame}

\frametitle{Convergent or not}

Context: Bush and noone else is on TV.

\ex. \a. Is Bush or Janet Jackson on TV? \b. True answer (K): Bush
is on TV. \c. True exhaustive answer (GS): Bush is on TV and Janet
Jackson is not on TV.\pause

\ex. \a. Is Bush or Will Ferrell on TV? \b. True answer (K): Bush
is on TV. \c. True exhaustive answer (GS): Bush is on TV and
Ferrell is not on TV.

\end{frame}

%%%%%%%%%%%%%%%%%%%%%%%

\begin{frame}

\frametitle{Comparison}

\begin{itemize}

\item on K.'s analysis: the two questions are convergent, and (A)
holds.

\item on GS's analysis: the two questions are not convergent, (A)
does not hold.

\end{itemize}

\ex. John knows that Bush is on TV and that Jackson is not

\ex. John knows that Bush is on TV and that Ferrell is not.


\end{frame}

%%%%%%%%%%%%%%%%%

\begin{frame}

\frametitle{Summary}

\begin{itemize}

\item Karttunen's semantics is consistent with Schaffer's
predictions, but too weak to be adequate for knowledge
attributions\pause

\item GS's semantics does not support Schaffer's predictions\pause

\item In both cases, we ignored further restrictions on the space
of answers: for instance, it may be presupposed that exactly one
person is on TV.

\end{itemize}




\end{frame}


\section{Context and Attributions}

\begin{frame}

\frametitle{A closer look at Schaffer's examples}

\begin{itemize}

\item Schaffer's intuition: knowing whether Bush or Janet Jackson is
on TV is easier than knowing whether Bush or Ferrell is on
TV.\pause

\item Contextualist idea: knowing whether A or B can be easier than
knowing whether A or C, if the alternatives provide suitable
\alert{restrictions} of the agent's epistemic state.\pause

\end{itemize}

\ex. ``S knows that P" is true in $w$ iff $K_w \subseteq P$\pause

\ex. \label{wh} ``S knows whether A or B" is true in $w$ iff
$K_{w}\cap(A \cup B) \subseteq A$ or $K_{w}\cap(A \cup B)
\subseteq B$



\end{frame}

%%%%%%%%%%%%%%%%%%%%%%%%%%%

\begin{frame}

\begin{figure}[t]


\centering{


\begin{tabular}{|c|}
\hline $\emptyset$\\ \hline
$B^{*}$\\
\hline
$F^{*}$\\
\hline
$J^{*}$\\
\hline
$(BF)^{*}$\\
\hline
$(BJ)^{*}$\\
\hline
$(FJ)^{*}$\\
\hline
$(BFJ)^{*}$\\
\hline
\end{tabular}}

\setlength{\unitlength}{0.5cm}

\begin{picture}(5,3)(1.7,-8)

\put(5,1){\oval(0.6,1.6)}

\put(4.9,1.3){\tiny{$w$}}

\put(5.2,0.8){\line(1,1){1}}

\put(6.3,1.8){$K_w$}

\end{picture}

\vspace{-1.3cm}

\caption{Who is on TV?}

\end{figure}
 \pause

\ex. \a. $K_{w} \cap (B \cup F) \nsubseteq B$, $\nsubseteq F$. \b.
Bill does not know whether Bush or Ferrell is on TV.\pause

\ex. \a. $K_{w} \cap (B \cup J) \subseteq B$ \b. Bill knows
whether Bush or Jackson is on TV.

\end{frame}

%%%%%%%%%%%%%%%


\begin{frame}

\frametitle{Partial answers}

\begin{itemize}

\item Suppose Bill is asked whether Bush or Jackson is on TV. Bill
is certain it is not Jackson, but thinks it might be Ferrell. What
would Bill answer to:

\ex. Is it Bush or Janet Jackson?

\ex. \a. (?) It's Bush. \b. At any rate, it's not Janet
Jackson.\pause

\item Another example (from Schaffer)

\ex. Was the thirteenth president of the US Millard Fillmore or
Hillary Clinton?

\ex. \a. (?) Millard Fillmore. \b. At any rate, not Hillary
Clinton.

\end{itemize}


\end{frame}

%%%%%%%%%%%%%%%%%

\begin{frame}

\frametitle{Dynamics of questions}

\ex. Is it a zebra or a horse?

\ex. A zebra.

\ex. Is it a zebra or a cleverly painted mule?

\ex. I don't know.

\ex. So is it a zebra, or a horse?

\ex. Well, not a horse, but...

\alert{Consequence}: one cannot take the alternatives present in
the question to systematically restrict the agent's epistemic
state.


\end{frame}

%%%%%%%%%%%%%%%%%%%

\begin{frame}

\frametitle{Alternative questions revisited}

\begin{tabular}{ll}

\small{Representation} & Q=$?(\phi \vee_{a} \psi)$= $?p ( p \wedge(p=\phi \vee p=\psi))$\\

\small{Denotation} & \val{Q}=$\{ (p, w) \mid w\in p \ \& \ p=
\val{
$\phi$} \ or \ p= \val{$\psi$} \}$\\

\small{Partition} & Part(Q)=$\{(w,v)\mid (p,w) \in \val{$Q$}\ iff\
(p,v)\in \val{$Q$}\}$\\

\small{Topics} & Top(Q)=$\{ p\mid \exists w: (p,w)\in
\val{$Q$}\}$\\

\end{tabular}

\ex. Did John leave, or did Mary leave? \a. Representation $?p ( p
\wedge (p=\phi \vee p=\psi))$ \b. Partition: $\{\phi \wedge \neg
\psi, \neg \phi \wedge   \psi, \neg (\phi\vee \psi), \phi \wedge
\psi\}$ \b. Topics: $\{\phi,\psi\}$

\end{frame}

%%%%%%%%%%%%%%%%%%%%

\begin{frame}

\frametitle{Contexts and Updates}

\bit

\item Context $C=(s_{C},i_{C})$: $s_{C}$=set of worlds(=context set); $i_{C}$=
sequence of question denotations (=issues under discussion).

\ex. \a. $C+P = (s_{C} \cap \val{$P$}, i_{C})$ \b. $C+Q= (s_C,
i_{C} + \val{$Q$})$

\item Assertions update the context set; questions update the
issues under discussion.

\eit

\end{frame}

%%%%%%%%%%%%%%

\begin{frame}

\frametitle{Knowledge contextualized}

Top($C$)= union of all Top(Q) for all Q in $C$.

\ex. ``S knows Q'' is true in world $w$ with respect to context
$C$ iff $K_w \ \cap\ $Top$(C) \ \subseteq \ $ANS$_w(Q)$
[simplified truth conditions]

\end{frame}

%%%%%%%%%

\begin{frame}

\frametitle{Back to Schaffer's example}

%\documentclass{article}

%\begin{document}

\vspace{0.2cm}

\begin{figure}[h]

%\vspace{.5cm}

\setlength{\unitlength}{1.7cm}

\begin{picture}(4,2)

\put(3,2){\line(1,0){2}}

\put(3,1){\line(1,0){2}}

\put(3,0){\line(1,0){2}}

%%%%%%%

\put(3,0){\line(0,1){2}}

\put(4,0){\line(0,1){2}}

\put(5,0){\line(0,1){2}}

%%%%%%

\put(3,0){\line(1,1){2}}

\put(3,1){\line(1,1){1}}

\put(4,0){\line(1,1){1}}

%%%%%%%

\put(2.7,1.5){$J$}

\put(2.5,0.5){$\neg J$}

\put(3.5,2.2){$B$}

\put(4.3,2.2){$\neg B$}

\put(3.75,1.1){$F$}

\put(4.75,1.1){$F$}

\put(3.75,0.1){$F$}

\put(4.75,0.1){$F$}

%%%%%%%%%%

\qbezier(3.3,0.3)(3.3,0.9)(3.7,0.7)

\qbezier(4.3,0.3)(4.7,0.1)(4.7,0.7)

\put(3.45,0.6){\tiny{$w$}}

%\put(4.45,0.3){\tiny{$v$}}


\end{picture}

\caption{Is it Bush, or is it Janet Jackson?}%{Another look at Figure 1}

\end{figure}

%\end{document}


\ex.\label{bj} \a. S knows whether it is Bush or Janet Jackson on
TV. \b. true in $C + ?(B \vee_a J)$, but false in $C +  ?(B \vee_a
J) + ?(B \vee_a F)$

\ex.\label{bf} \a. S knows whether it is Bush or Ferrell on TV.
\b. false in $C + ?(B \vee_a F)$, and likewise false in $C + ?(B
\vee_a F) + ?(B\vee_a J)$.


\end{frame}


\section{Presupposition}

\begin{frame}

\frametitle{Presupposition failure}

Context: I know I did not leave my keys in the fridge. Not sure
about where I left them (sofa or table):

\ex. I don't know whether I left my keys on the sofa or by the
table

\ex. I don't know whether I left my keys on the sofa or in the
fridge\pause

\bit

\item Is the second sentence false, or undefined?\pause

\item Our prediction: both sentences are true relative to $?(S\vee
T)+?(S+F)$; asymmetry relative to $?(S+F)$.

\eit

\end{frame}

\begin{frame}

\frametitle{The negation problem}

\ex. Bill does not know whether A or B

\ex. Bill does not know whether A and Bill does not know whether B

\begin{itemize}\pause

\item Solution 1: say that ``S knows whether A or B" presupposes
``S knows exactly one of A or B to be true". (too strong in our
opinion)\pause

\item Solution 2: say that ``S knows whether A or B" presupposes
``S knows exactly one of A or B to be true, or knows both or
neither to be true" (our proposal)

\end{itemize}\pause

Problem: source of the presupposition? (disjunction? symmetry
(Chemla)?)

\end{frame}

%%%%%%%%%%%%%%%


\section{Conclusion}

Schaffer's examples make a new case for the context-sensitivity of
knowledge ascriptions. While we agree on the contextualist
conclusion Schaffer draws from his examples, we nevertheless
disagree with him on several points of details, and more
substantially, on the extent given to the contextualist analysis.
In the first section of this paper, we have argued that the schema
of convergent knowledge does not provide a straightforward
argument against the reductive analysis of knowledge-\emph{wh} to
knowledge-\emph{that}, in particular if one adopts the partition
theory of questions. In the last part, we have moreover argued
that some of the inequivalences observed by Schaffer should be
traced to the presuppositions associated with the use of
\emph{whether}-complements, rather than to the assertive content
they contribute under the scope of attitude verbs. More
fundamentally, on our account Schaffer's examples are revealing of
a form of ambiguity in what ``knowing the answer" means, roughly
between knowing the true exhaustive answer prior the question is
asked, and being able to tell the answer if the question were
asked explicitly (relying on knowledge of the partial answer).
Like Schaffer, however, we do agree that the meaning of questions
is dependent both on the domain of quantification and on the
alternatives that are considered relevant. But this is not
sufficient to conclude, as Schaffer does, that
knowledge-\emph{that}, no more than knowledge-\emph{wh},
systematically ``includes a question". On the present view,
``know" continues to denote a binary relation between an agent and
a proposition, even when the context is more richly articulated.



\end{document}
