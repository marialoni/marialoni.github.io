In a recent paper, Jonathan Schaffer presents an argument against
the received view amongst epistemologists that knowledge-\emph{wh}
ascriptions can be analyzed in terms of knowledge-\emph{that}
ascriptions. On the reductive analysis of knowledge-\emph{wh}, to
know $Q$ in a context $w$, where $Q$ is an embedded question, is
to know the true answer to $Q$ in $w$. The reductive analysis
makes the prediction that if two questions $Q$ and $Q'$ have the
same answer $P$ in a given context, then knowing $Q$ and knowing
$Q'$ should come out semantically equivalent: thus, convergent
questions are predicted to yield equivalent ascriptions when
embedded under \emph{know}. Schaffer objects to that prediction,
by pointing out that in a context in which Bush is on TV, I can
easily know whether Bush or Janet Jackson is on TV (because
distinguishing between the two is easy), and yet fail to know
whether Bush or Will Ferrell is on TV (because Ferrell is such a
good impersonator of Bush). Thus, although the two questions ``is
Bush or Janet Jackson on TV?" and ``is Bush or Will Ferrell on
TV?" have the same actual answer (namely ``Bush is on TV"), they
do not yield equivalent knowledge claims. On the basis of this and
related examples, Schaffer proposes to abandon the reductive
analysis, and at the same time to defend the opposite view that
knowledge-\emph{that} claims themselves are relative to questions.

According to Schaffer, the reductive analysis stems from two main
sources: the first is the assumption that
\emph{knowing}-\emph{that} constructions are more primitive than
other constructions involving \emph{know}; the second, according
to him, is the idea that \emph{know} denotes a binary relation
between an agent and a proposition.
%On Schaffer's account, on the contrary, a knowledge-that
%ascription is in fact always relative to a question, and know
%should therefore denote a ternary relation between agents,
%propositions and questions.
In linguistic circles, however, it may be pointed out that the
reductive analysis of knowing-\emph{wh} in terms of
knowing-\emph{that} is also the dominant view, but primarily
because the meaning of (unembedded as well as embedded) questions
is itself conceived reductively in terms of the meaning of their
answers. On Karttunen's semantics for questions, for instance, a
question $Q$ in a context $w$ denotes the set of all true
propositional answers to $Q$ in $w$. On Groenendijk and Stokhof's
account, a question $Q$ denotes a function which to each world $w$
associates the true and exhaustive answer to the question $Q$ in
$w$. Although the two theories differ, in both of them the
reductive analysis of knowing-\emph{wh} in terms of
knowing-\emph{that} is itself parasitic on the reductive analysis
of the meaning of questions in terms of their propositional
answers, together with the principle of compositionality. Thus, if
we let \val{$Q$} stand for the meaning or intension of $Q$, and
\val{$Q$}(w) the value of $Q$ in $w$, namely the true
propositional answer to $Q$ in $w$, it may be checked that both
Karttunen's semantics and Groenendijk and Stokhof's semantics
satisfy the prediction that convergent questions yield equivalent
knowledge-wh claims, namely:\footnote{$K$ stands to ``know" and
$X$ names an agent. The schema (CV) follows in both theories from
the assumption of compositionality by which \val{$X K
Q$}($w$)=\val{$KQ$}($w$)(\val{$X$}($w$)), and
\val{$KQ$}($w$)=$\lambda x.$\val{$K$}($w$)\val{$Q$}($w$). The
difference is that in Karttunen's semantics, a question is an
entity of type $s((st)t)$, while in Groenendijk and Stokhof's
semantics it is of type $s(st)$; in Groenendijk and Stokhof's
semantics, \emph{know} is of uniform type $(st)(et)$; in
Karttunen's semantics, the question-embedding variant of
\emph{know} is of type $((st)t)(et)$.}

\begin{example}[(CV)]

\begin{tabular}{l}

\val{$Q$}($w$)\ =\val{$Q'$}($w$)\\

\hline

\val{$X K Q$}($w$)\ =\val{$X K Q'$}($w$)

\end{tabular}

\end{example}


Our aim in this paper is to challenge Schaffer's analysis of his
own counterexample to (CV), and to defend the reductive analysis
of knowing-\emph{wh} to knowing-\emph{that} (and with it the
reductive analysis of questions in terms of the meaning of their
answers)\footnote{As it turns out, objections similar to
Schaffer's objection to (CV) have been raised against Groenendijk
and Stokhof's theory of questions specifically. In particular,
Groenendijk and Stokhof's theory of questions identifies the
intension and extension of the two questions: ``who left?" and
``who did not leave?". Arguably, however, one can know who left
without knowing who did not leave. This objection, however, is not
an objection against (CV), but rather an objection to a particular
way of individuating questions (as involving strongly exhaustive
answers). The example, incidentally, is more convincing with verbs
like ``being surprised about", for which weakly exhaustive
readings are more clearly attested (see Heim 1994 and Sharvit
2002). We do agree with Groenendijk and Stokhof that questions
after \emph{know}, unlike after \emph{surprise}, have strongly
exhaustive readings, but this point is orthogonal to the main
issue here.}. Schaffer's proposed counterexample to (CV) involves
two alternative questions of the form ``whether A or B" vs
``whether A or C". In the first of the paper, we propose a closer
analysis of the meaning of alternative questions, and argue that
Schaffer's example is in fact not a real counterexample to (CV):
we examine several ways of understanding Schaffer's proposal, and
show that on all of them, (CV) either remains sound, or simply
fails to be correctly instantiated. In the second part of the
paper, we argue that there remains a sense in which Schaffer's
intuition is correct nevertheless: namely, one ought to
distinguish ``knowing whether A or B" from knowing the answer to
the question ``is A or B the case?". More precisely, there may be
two ways of understanding ``knowing $Q$": before the question is
asked explicitly, and after it is asked explicitly. The second
sense, however, is weaker that he first, since it rests on some
form of presupposition accommodation.
