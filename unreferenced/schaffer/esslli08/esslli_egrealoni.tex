
%
%  Guidlines for
%
%  $Author: ienne $
%  $Date: 1995/09/15 15:20:59 $
%  $Revision: 1.4 $
%

\documentclass[times,10pt,twocolumn]{article}
\usepackage{latex8}
\usepackage{times}



%-------------------------------------------------------------------------
% theorem,definition, etc. environments.
%
%
%The following packages are used.    Email epacuit@stanford.edu
%if there is a problem.
\usepackage{amsmath,amsfonts,latexsym,amssymb,url,theorem,color}

%%theorem environments
\newtheorem{theorem}{Theorem}[section]
\newtheorem{proposition}[theorem]{Proposition}
\newtheorem{lemma}[theorem]{Lemma}
\newtheorem{corollary}[theorem]{Corollary}

%%proof environment
\newenvironment{proof}{\begin{trivlist}\item[]{\bf
Proof.}}{\hfill {\sc qed}\end{trivlist}}
\newenvironment{proofof}[1]{\begin{trivlist}\item[\hskip\labelsep{\bf
Proof~of~{#1}.\ }]}{\hspace*{\fill} {{\sc qed }(of Claim)}
\end{trivlist}}

%%for definitions
{\theorembodyfont{\rmfamily}
\newtheorem{defi}[theorem]{Definition}}

\newenvironment{definition}{\begin{defi}\rm}{\hfill
$\triangleleft$\end{defi}}

\newenvironment{Definition}[1]{\begin{defi}[{#1}]\rm}{\hfill
$\triangleleft$\end{defi}}


\newtheorem{fact}[theorem]{Fact}
\newtheorem{conjecture}[theorem]{Conjecture}
\newtheorem{convention}[theorem]{Convention}
\newtheorem{question}[theorem]{Question}
\newtheorem{exercise}[theorem]{Exercise}
\newtheorem{remark}[theorem]{Remark}
\newtheorem{exam}[theorem]{Example}
\newtheorem{openprob}[theorem]{Open Problem}
\newtheorem{observation}[theorem]{Observation} %%niet in LLNCS style

\newenvironment{example}[1]{\begin{exam}[{#1}]\rm}{\end{exam}}
\newenvironment{openproblem}[1]{\begin{openprob}[{#1}]\rm}{\end{openprob}}


%
%-------------------------------------------------------------------------


%-------------------------------------------------------------------------
\usepackage{linguex}

\newcommand{\bl}{$\bullet$\ }
\newcommand{\val}[1]{\mbox{ $[\![$  #1 $]\!]$}}
\newcommand{\bit}{\begin{itemize}} \newcommand{\eit}{\end{itemize}}
\newcommand{\ben}{\begin{enumerate}} \newcommand{\een}{\end{enumerate}}
\newcommand\notext[1]{}
\newcommand{\lin}{\ensuremath{\lbrack\!\lbrack}}
\newcommand{\rin}{\ensuremath{\rbrack\!\rbrack}}

%
%-------------------------------------------------------------------------


%-------------------------------------------------------------------------
% do not remove --- page numbers will be added later
\pagestyle{empty}

%-------------------------------------------------------------------------
\begin{document}

\title{Knowing whether A or B}

\author{Maria Aloni\\
ILLC, Amsterdam \\M.D.Aloni@uva.nl\\
% For a paper whose authors are all at the same institution,
% omit the following lines up until the closing ``}''.
% Additional authors and addresses can be added with ``\and'',
% just like the second author.
\and
Paul \'Egr\'e   \\
Institut Jean-Nicod, Paris\footnote{EHESS/ENS/CNRS.}\\
paulegre@gmail.com\\
}

\maketitle \thispagestyle{empty}

\begin{abstract}

Can we say that $s$ knows whether A or B when $s$ is only able to
rule out A, but remains uncertain about B? We discuss a set of
examples put forward by J. Schaffer's in favour of a contextualist
answer to this problem. We present a context-sensitive and dynamic
semantics for knowledge attributions, in which those can depend on
the alternatives raised by the embedded question, but also on
alternatives raised earlier in the context.

\end{abstract}


\Section{Alternative questions in epistemic contexts}

The aim of this paper is to discuss the semantics of knowledge
attributions of the form ``$s$ knows whether A or B", which we may
symbolize by $K_{s} ?(A\vee_{a} B)$, where $?(A \vee_{a} B)$
denotes an alternative disjunctive question, like ``is John in
London, or is Mary in London?". More specifically, our aim is to
provide a dynamic account of the context-sensitivity of such
attributions.

It is standard in linguistic theory to distinguish polar readings
and alternative readings of disjunctive questions (see e.g.
Haspelmath 2000, Han and Romero 2003). Under the polar reading, a
question of the form ``is John or Mary in London?" calls for a yes
or no answer. The polar reading can be forced in English by asking
``is either John or Mary in London?". For the alternative reading,
by contrast, the question cannot be answered by yes or no and has
to be answered by a sentence like ``John is London", or ``Mary is
not in London", namely by providing information about the truth
and falsity of the respective disjuncts.

There is still some debate in the literature about the answerhood
conditions of alternative questions, and by way of consequence,
about the conditions under which a subject can be said to know
whether A or B. In a recent paper (Schaffer 2007), J. Schaffer
argues that in a context in which $s$ sees someone on TV, who is
actually George Bush, but such that $s$ is not able to
discriminate between George Bush and Will Ferrell (because Ferrell
is such a good impersonator of Bush), and yet is able to see that
it is not Janet Jackson, \ref{fer} below should be judged false,
but \ref{jack} should count as true:

\ex. \a.\label{fer} $s$ knows whether George Bush or Will Ferrell
is on TV \b.\label{jack} $s$ knows whether George Bush or Janet
Jackson is on TV.

The intuition reason for the truth of \ref{jack}, according to
Schaffer, is that the question ``is Bush or Janet Jackson on TV?"
is easier for $s$ to answer than the question ``is Bush or Will
Ferrell on TV?". In our view, however, ordinary intuitions are
less stable: although \ref{fer} should be incontrovertibly false
in the scenario, the status of \ref{jack} is much less clear. In
our opinion, all that $s$ really knows is that \emph{Janet Jackson
is not on TV}, which need not be sufficient to fully answer the
question ``is Bush or Janet Jackson on TV?".

More formally, assuming the partition theory of questions of
Groenendijk and Stokhof (1984), an answer of the form ``Janet
Jackson is not on TV" counts only as a \emph{partial answer} to
the question ``is Bush or Janet Jackson on TV?". For $s$ to know
the complete answer to the question ``is Bush or Janet Jackson on
TV", $s$ should know more, namely that Bush is on TV and that
Janet Jackson is not on TV. The partial answer ``Janet Jackson is
not on TV" would count as complete if one presupposed that exactly
one of the two disjuncts had to be true. In principle, however,
there is no more reason to think that ``$s$ knows whether Bush or
Janet Jackson is on TV" is true than there is to think that ``$s$
knows whether Ferrell or Janet Jackson is on TV" is true. In other
words, $s$'s ignorance about who exactly \emph{is} on TV seems to
override $s$'s partial knowledge about who is \emph{not} on TV.

Despite this, we agree with Schaffer that there is a sense in
which, if $s$ is allowed to \emph{ignore} the possibility that
Ferrell might be on TV, then $s$ can be said to know whether Bush
or Janet Jackson is on TV, simply based on $s$'s knowledge of that
partial answer.

\Section{Dynamics of knowledge attributions}

To implement this idea, we propose a question semantics for
knowledge in which attributions involving questions can be made
sensitive both to the alternatives raised by the question, as well
as to alternatives raised earlier in the context. The semantics is
dynamic, in so far as the context can be incremented with the
considerations of new alternatives, in a way that not simply
restricts, but can also increase, the subject's uncertainty.
% Here we present the restriction of the system to
%alternative questions, but the actual system deals with polar
%questions and with more complexe constituent questions.

\SubSection{Question semantics}

Questions in the system are represented by formulas of the form $?
p_1, ..., p_n \ \phi$ where ? is a query-operator, $p_1, ..., p_n$
is a possibly empty sequence of propositional variables, and
$\phi$ is a formula of predicate logic with propositional
variables. In the case of alternative questions, a question of the
form ``is $\phi$ or $\psi$?" (abbreviated $?(\phi \vee_{a} \psi)$)
is represented by a formula of the form $?p ( p \wedge (p=\phi
\vee p=\psi))$, which asks which of the propositions $\phi$ and
$\psi$ is true.

Questions denotations are then defined as follows, where $\vec{p}$
stands for the sequence $p_1,...,p_n$, and  $\vec{\alpha}$ for
 the sequence $\alpha_1,...,\alpha_n$: $\lin? \vec{p} \ \phi\rin_{M,g} = \{ \langle
\vec{\alpha}, w\rangle \mid  \ w \in  \lin  \phi\rin_{M,g[\vec{p}
/ \vec{\alpha}  ]}  \}$. The denotation of an alternative question
$?p (p \wedge    (p=\phi \vee p=\psi))$ is thus the set of pairs
$\langle  p, w \rangle$ such that $w$ satisfies $p$ and  $p$ is
either the proposition expressed by $ \phi $ or the proposition
expressed by $  \psi $. From the denotation of a question, we can
define the \emph{partition} Part$(?\vec{p}  \ \phi)$ induced by
the question $?\vec{p}  \ \phi$ as the set of ordered pairs
$\langle w,v\rangle$ such that for all proposition $\vec{\alpha}$,
$\langle \vec{\alpha}, w\rangle \in \lin ? \vec{p} \ \phi
\rin_{M,g}$ iff $\langle \alpha, v\rangle \in \lin ? \vec{p} \
\phi \rin_{M,g}$. Finally, we define the \emph{topics} raised by a
question as the set Top$_{M,g}(?\vec{p} \ \phi)$ = $\{
\vec{\alpha}  \mid \exists w: \langle  \vec{\alpha} , w\rangle \in
\lin ?\vec{p}  \ \phi\rin_{M,g} \}$. For alternative questions,
one can check that Part($?(\phi \vee_{a} \psi$))=$\{\phi \wedge
\neg \psi, \neg \phi \wedge \psi, \neg \phi \wedge \neg \psi, \phi
\wedge \psi\}$, and Top$(?(\phi \vee_{a} \psi))$=$\{\phi, \psi\}$.

\SubSection{Knowledge and context updates}

A \emph{context} $C$ is defined as an ordered pair whose first
index $s_C$ is an information state (set of worlds), and whose
second index $i_C$ is a sequence of question denotations
representing  the issues  under discussion in $C$. A context $C$
can be updated either by an assertion $P$, or by the introduction
of a new question $Q$:

\ex. \a. $C+P = (s_C \cap \lin P\rin, i_C)$ \b. $C+Q= (s_C,
i_C+\lin Q\rin)$

We let ANS$_w(Q)$ be the true exhaustive answer to $Q$ in $w$ (the
cell containing $w$ in Part($Q$), and Top$(C)$ denote the union of
the topics introduced by all the issues in $C$, i.e. for $C=
(s_c,\lin Q_1 \rin, ..., \lin Q_n \rin)$: Top$(C)$ =
$\bigcup_{i\in n} $Top$(Q_i) \setminus \{\langle \rangle\}$.

Define $\mathcal{K}_{s}(w)$ to be the knowledge state of $s$ in
$w$, namely the set of epistemically accessible worlds to $s$. We
then define knowledge as follows:

\ex. ``$s$ knows Q'' is true in world $w$ with respect to context
$C$ iff \a.[(i)] $\mathcal{K}_{s}(w) \ \cap\ $Top$(C) \ \subseteq
\ $ANS$_w(Q)$, \ \ if  Top$(C)\neq\emptyset$; \b.[(ii)]
$\mathcal{K}_{s}(w) \ \subseteq \ $ANS$_w(Q)$,  \ \ otherwise.


\Section{Schaffer's puzzle}

Going back to Schaffer's example, suppose $\mathcal{K}_{s}(w)$ is
a state compatible with Bush being on TV ($B$) and with Ferrell
being on TV ($F$), but excluding Janet Jackson being on TV ($J$).
The following holds:

\ex.\label{bj} \a. S knows whether it is Bush or Janet Jackson on
TV. \b. true in $C + ?(B \vee_a J)$, but false in $C +  ? (B
\vee_a J) + ? (B \vee_a F)$

\ex.\label{bf} \a. S knows whether it is Bush or Ferrell on TV.
\b. false in $C + ?(B \vee_a F)$, and likewise false in $C + ?(B
\vee_a F) + ?(B\vee_a J)$.

The semantics predicts that when $s$'s knowledge state is
restricted to the topics raised by ``is Janet Jackson or Bush on
TV?", $s$ will know the answer. But if a further issue comes up
after this question was asked, namely ``is Bush or Ferrell on
TV?", then $s$ may no longer be said to know whether Bush or Janet
Jackson is on TV, because the context is incremented with a third
alternative (namely the possibility that it might be Ferrell).

%%%%%%%%%%%

\Section{Perspectives}



The semantics here presented can be used to deal with other
scenarios involving, in particular, the consideration of skeptical
alternatives, whereby the introduction of a new alternative can
impair one's initial confidence in the particular answer to a
question. We shall explain the extension of the semantics to other
types of questions, and discuss possible connections with the
topic of unawareness. A further issue, which we elaborate in the
paper, concerns the partialization of the semantics, to deal with
presupposition failure. Thus, in a situation in which $s$ holds a
partial answer to the question, as in Schaffer's scenario, the
negation of \ref{jack} may be judged inappropriate, hence neither
true nor false, rather than true at all. The partiality can be
derived from the assumption that $s$'s uncertainty should always
be symmetric with respect to the alternatives raised by the
question.

\begin{thebibliography}{0}

\bibitem{Aloni} Aloni M. \& \'Egr\'e P. (2008), ``Alternative
Questions and Knowledge Attributions", manuscript, under review.

\bibitem{GS} Groenendijk, J. and Stokhof M. (1984), \emph{Studies
in the Semantics of Questions and the Pragmatics of Answers}. PhD
dissertation, University of Amsterdam.


\bibitem{Hasp} Haspelmath, M. (2000),
``Coordination". To appear in: T. Shopen (ed.) \emph{Language
typology and syntactic description}. 2nd ed. Cambridge: Cambridge
University Press.

\bibitem{chan} Romero, M. and Han C-H. (2003), ``Focus, Ellipsis and
the Semantics of Alternative Questions", in \emph{Empirical Issues
in Formal Syntax and Semantics 4}. C. Beyssade, O. Bonami, P.
Cabredo Hofherr, F. Corblin (eds), Presses Universitaires de
Paris-Sorbonne, Paris, 291-307.

\bibitem{Schaffer} Schaffer, J. (2007), ``Knowing the Answer",
\emph{Philosophy and Phenomenological Research}, Vol. LXXV No. 2,
1-21.

\end{thebibliography}











\end{document}
