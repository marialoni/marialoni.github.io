 \section*{Appendix}

In this appendix we present a dynamic semantics for questions and
their embedding under ``know". The system is one way of
formalizing Schaffer's remarks on the dynamics of conversation,
but making explicit the importance of the order in which questions
are
asked. %We do not integrate issues about presupposition here.

 %\begin{itemize}
 %\item

 \paragraph{Questions denotations.}
 %(more or less Hamblin (possible answers rather than true answers):

Question denotations are defined Hamblin-style, namely as the set
of possible answers (rather than true answers) to the question. To
mark the disjunction in alternative questions, we use $\vee_a$. A
polar question like ``did John leave?" denotes the set $\{$John
left, John did not leave$\}$; an alternative question like ``did
John leave, or did Mary leave?" denotes the set $\{$Mary left,
John left$\}$; a constituent question like ``who left?" denotes
the set of its elementary answers over the domain of individuals,
e.g. $\{$John left, Mary left, Susan left, ...$\}$.

 \ex. Polar questions (``did John leave?")
 \a. syntax: $?\phi$
 \b. semantics: $\lambda p  [p=\phi \vee p=\neg \phi]$ \hfill $\{\phi, \neg \phi\}$

 \ex. Alternative questions (``did John leave, or did Mary leave?")
  \a. syntax: $?(\phi  \vee_a \psi)$
 \b. semantics: $\lambda p  [p=\phi \vee p= \psi]$ \hfill $\{\phi,   \psi\}$

\ex. Constituent questions (``who left?")
 \a. syntax: $?x \phi   $
 \b. semantics: $\lambda p  [\exists x (p=\phi(x))]$ \hfill $\{\phi(d),   \phi(d'),...\}$

%%%%%%%%%%%%%%%%%%%

\paragraph{Answers.} The denotation $Q$ of a question Q determines
an equivalence relation (or, equivalently, a partition) Part($Q$)
over the set of possible worlds, from which different notions of
answers can be defined. Two worlds $w$ and $v$ are in the same
cell of the partition if they belong to exactly the same members
of the denotation of Q.

\ex. Part($Q$) = $\{(w,v) \mid \forall \alpha \in Q: w \in \alpha
\ \ \mbox{iff} \ \ v \in \alpha \}$


\noindent \emph{Exhaustive answers} to $Q$ correspond to cells in
Part($Q$). The \emph{exhaustive true answer} to $Q$ in $w$ is the
cell including $w$ in Part($Q$). Finally, \emph{partial answers}
(true in $w$) correspond to non-trivial unions of cells of
Part($Q$) (including $w$), namely unions of cells different from
Part($Q$).

\paragraph{Context and Updates.} %(this is very crude):

%\begin{itemize}

%\item with assertions

A context $C$ is defined as an ordered pair whose first index
$s_C$ is an information state (set of worlds), and whose second
index $QUD_C$ is a question under discussion (set of
propositions). A context $C$ can be updated either by an assertion
$P$, or by the introduction of a new question $Q$:

\ex. \a. $ C + P = (s_C \cap  P, QUD_{C})$\quad (assertions) \b.
$C + Q = (s_C, QUD_{C} \cup Q)$\quad (questions)

%\item Without assertions:

%\ex.   $C + Q = (C \cup Q)$

%\end{itemize}

\paragraph{Knowledge.} Knowledge ascriptions are evaluated relative
to a possible world and a context. A subject S knows Q if S's
belief state, when restricted to the alternatives raised by the
question under discussion, entails the exhaustive true answer to
Q.

\ex.  ``S knows Q" is true wrt $C$ in $w$ iff $K_w \cap (\cup\
QUD_C)$ entails the exhaustive true answer to Q in $w$

%Or alternatively (prob equivalent) \ex. s knows Q wrt C in w iff
%for all true partial answer $A$ $(K_w \cap QUD_C)$ entails A in w

%(this second formulation would maybe allow us  to express partial
%knowledge, and the following entailment (if we define not know $Q$
%= for all  partial answers A not know (A)) :

%\ex. not know  ?(A or B) $\Rightarrow$ not know ?A and not know  ?B

\paragraph{Schaffer's example.} Let us take $s_C$ to be $\{w_{B^{*}}\}$, namely
the actual world $w=w_{B^{*}}$ is a world in which only Bush is on
TV; $K_w= \{w_{B^{*}}, w_{F^{*}}\}$, that is S's belief state is
compatible with either only Bush being on TV, and only Ferrell
being on TV. When the dialogue starts, {\bf QUD$_C$}  is empty. The semantics
derives the following predictions:

\ex. \a. S knows whether it is Bush or Janet Jackson on TV. \b.
true in $C + ?(B \vee_a J)$, but false in $C +  ? (B \vee_a J) + ?
(B \vee_a F)$

\ex. \a. S knows whether it is Bush or Ferrell on TV. \b. false in
$C + ?(B \vee_a F)$, and likewise false in $C + ?(B \vee_a F) +
?(B\vee_a J)$.

\enlargethispage{1cm}

%\documentclass{article}

%\begin{document}

\vspace{0.2cm}

\begin{figure}[h]

%\vspace{.5cm}

\setlength{\unitlength}{1.7cm}

\begin{picture}(4,2)

\put(3,2){\line(1,0){2}}

\put(3,1){\line(1,0){2}}

\put(3,0){\line(1,0){2}}

%%%%%%%

\put(3,0){\line(0,1){2}}

\put(4,0){\line(0,1){2}}

\put(5,0){\line(0,1){2}}

%%%%%%

\put(3,0){\line(1,1){2}}

\put(3,1){\line(1,1){1}}

\put(4,0){\line(1,1){1}}

%%%%%%%

\put(2.7,1.5){$J$}

\put(2.5,0.5){$\neg J$}

\put(3.5,2.2){$B$}

\put(4.3,2.2){$\neg B$}

\put(3.75,1.1){$F$}

\put(4.75,1.1){$F$}

\put(3.75,0.1){$F$}

\put(4.75,0.1){$F$}

%%%%%%%%%%

\qbezier(3.3,0.3)(3.3,0.9)(3.7,0.7)

\qbezier(4.3,0.3)(4.7,0.1)(4.7,0.7)

\put(3.45,0.6){\tiny{$w$}}

%\put(4.45,0.3){\tiny{$v$}}


\end{picture}

\caption{Is it Bush, or is it Janet Jackson?}%{Another look at Figure 1}

\end{figure}

%\end{document}


Figure 2 is another, more adequate way of looking at Figure 1 in
order to represent the question ``is it Bush, or is it Janet
Jackson?". Each cell in the partition induced by $?(B\vee_{a} J)$
is itself partitioned into $F$-worlds (below the diagonal) and
$\neg F$-worlds (above the diagonal). S's belief state $K_w$
consists of the curved portion containing $w$ above the diagonal
in the area $B\neg J \neg F$, together with the curved portion
below the diagonal in the area $\neg B \neg J F$.

Suppose S is asked first whether Bush or Janet Jackson is on TV.
This updates $C$ with the question under discussion $?(B\vee_a
J)$. Relative to that question, namely to the alternatives
$\{B,J\}$, S can give the true exhaustive answer ``Bush and not
Janet Jackson". Then $S$ is asked whether Bush or Ferrell is on
TV. This increments the context with a new question under
discussion, namely $?(B \vee_a F)$, which enlarges the space of
relevant alternatives to $\{B,J,F\}$. This time, S's state $K_w$
no longer supports the exhaustive answer ``Bush and not Jackson",
for it also overlaps with the exhaustive answer ``neither Bush nor
Jackson". In other words, S's belief state only supports the
partial answer ``not Jackson".

The notion of partial answer, it may be noted, remains consistent
with Schaffer's intuition that the question ``is it Bush, or Janet
Jackson?" is easier than the question ``is it Bush, or Ferrell?",
since S can at least provide a partial answer to the first.

The system predicts that ``S knows whether Bush or Ferrell is on
TV" will come out true if the question under discussion is simply
$?(B\vee_a J)$, namely ``is Bush or Janet Jackson on TV?". But
this will not happen: for the background question under discussion
should always include the question discussed in the knowledge
report.






%\ex. \a. Ralph knows whether it is Bush or Ferrell on TV. \b.
%false in $C+ ?(b \vee_A f)$


 % \ex. Ralph   knows  whether it is Bush.
 % \a. true if Ferrell $\not\in C$
 % \b. false if Ferrell $\in C$



%\end{itemize}
